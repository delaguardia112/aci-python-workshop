\documentclass[12pt]{article}

\usepackage{listings}
\usepackage{color}
\usepackage{amsmath}
\usepackage{graphicx}

\usepackage{lastpage}

\usepackage{fancyhdr} 
\pagestyle{fancy}
\fancyhf{}
\rhead{Sushobhan Sen}
\lhead{ACI-UIUC Python Workshop}
\cfoot{\thepage\ of \pageref{LastPage}}

\usepackage{hyperref}
\usepackage[
    type={CC},
    modifier={by-nc-sa},
    version={3.0},
]{doclicense}

\begin{document}

%%%----------------------------------
%Settings for typesetting Python Code
%%%----------------------------------
\definecolor{keywords}{RGB}{255,0,90}
\definecolor{comments}{RGB}{0,0,113}
\definecolor{red}{RGB}{160,0,0}
\definecolor{green}{RGB}{0,150,0}
 
\lstset{language=Python, 
        basicstyle=\ttfamily\small, 
        keywordstyle=\color{keywords},
        commentstyle=\color{comments},
        stringstyle=\color{red},
        showstringspaces=false,
        identifierstyle=\color{green}}
        
%%%----------------------------------

\begin{titlepage}
	\centering
	{\LARGE American Concrete Institute (ACI)-UIUC Student Chapter}
	\vspace{2cm}	
	
	\begin{Large}
	Python Workshop (Basic and Advanced)
	\vspace{1cm}
	
	\textbf{Sushobhan Sen}
	\vspace{1cm}
	
	Doctoral Candidate
	\vspace{1cm}
	
	April 6, 2019
	\vspace{1cm}
	
	Urbana, IL
	\end{Large}
\end{titlepage}

\doclicenseThis
\newpage

\tableofcontents
\newpage

\section{Introduction}
The Python programming language was conceived in the 1980s and the first implementation was deployed in 1991 by Guido van Rossum at Centrum Wiskunde \& Informatica (CWI) in the Netherlands. Python is designed to be a high-level, interpreted, general-purpose computer programming language with exception-handling and the ability to be extended by users. Crucially, Python is an open-source language, meaning anybody can see the code behind it. Python today is a very popular language is a broad variety of disciplines: machine learning, web development, Geographical Information Systems (GIS), engineering, database management, high performance computing, etc.

Python 2.0, which truly launched the popularity of the language, was released in 2000, with the last version in the series, Python 2.7, set to be phased out in 2020. The current recommended standard as of when I wrote this is Python 3.7.2, which is crucially \textbf{not backwards compatible} with Python 2.x. All subversions of Python 3.x are backwards compatible though, so feel free to update regularly.

(The above information was taken from \href{https://en.wikipedia.org/wiki/Python_(programming_language)}{Wikipedia})

The beauty of Python is its ease - the syntax is extremely simple as compared to many other popular programming languages. It is often said, not entirely without truth, that the difference between Python code and pseudo-code is merely the indentation. Thus, the language is very easy to learn for programming novices. However, its ease should not be misconstrued to mean that it is only for easy or trivial applications: Python is an extremely powerful language with a wide variety of applications, it has extensive documentation, and its open-source nature ensures that new features are being developed continuously while bugs are also being fixed.

\begin{lstlisting}[frame=single] 
for i in range(0,1):
	print(i)
\end{lstlisting}

\end{document}